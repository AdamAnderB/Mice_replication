\section{Abstract}

This study explores the mechanisms that drive speech sound acquisition, specifically focusing on how surprisal (i.e., prediction error) and cue competition facilitate learning, by replicating and extending \textcite{nixon2020mice}'s error-driven learning framework. We adapt the experiment to include Japanese and Mandarin, exploring how surprisal shapes learning across three types of speech contrasts (Southern Min Tone, Japanese short and long vowels, and Chinese fricatives). Additionally, the eye-tracking extension allows us to operationalize prediction error through participants’ gaze patterns during training, providing real-time data on how discrepancies between predicted and actual outcomes influence learning. In combination, these methods provide a nuanced view of the dynamic feedback loop between surprisal and learning in both testing and real-time processing during training, contributing to a deeper understanding of the adaptive processes underlying speech acquisition.

This study fits within the broader literature that challenges the view of language learning as a purely statistical tracking of speech cues. Instead, error-driven models, such as the Rescorla-Wagner model, suggest that learners use more than sheer accumulated frequency and co-occurrence in learning. That is, they use feedback from prediction error. \textcite{nixon2020mice} demonstrated that both cue competition and unlearning from prediction error are critical components of speech sound acquisition. Our results suggest that learning is not only driven by frequency but also by prediction error across speech contrasts. Further, the eye-fixation data suggests that less prediction-error may account for unacquired speech contrasts. By extending this paradigm to Japanese and Mandarin, this study provides new insights into cross-linguistic variation in the role of prediction error and frequency during second language speech learning. These findings offer important pedagogical insights, suggesting that instructional methods focusing on eliciting surprisal may not only enhance learners’ sensitivity to subtle speech contrasts, but may be necessary in teaching difficult-to-acquire sounds (e.g., Arabic pharyngeals, Korean stops, Japanese pitch-accent, Spanish taps/trills, and/or Mandarin tone) across different languages, optimizing learning outcomes in second language classrooms.

\section{Keywords}
Error-Driven Learning, Prediction Error, Cross-Linguistic Study, Japanese, Mandarin, Southern Min, Eye-Tracking
\newpage


