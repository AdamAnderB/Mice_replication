\section{2. Methods}

\subsection{Participants}
The recruitment of 180 Natives English speakers were managed through Prolific \citep{Palan_2018}. 60 participants were recruited for each language group. This number was determined through power analysis before recruitment began. The simulations for power analysis were based on the mixed effect model outcomes from the unlearning experiment being replicated here in \cite{nixon2020mice}. Results of the power analysis can be found, here.   For the purposes of this study, we define a native English speaker as one that is both English dominant and learned English as their first language \citep{Brown_Tusmagambet_Rahming_Tu_DeSalvo_Wiener_2023}. All participants were retained for language experience, while some had minor language exposure in high-school. The average age of participants was 

To ensure that our comparison to \cite{nixon2020mice}'s results were valid, we further specified that the participants were located in the United States. Nine additional participants outside the 180 collected participants were removed for low digit span scores. An additional 41 were removed for failed headphone-checks \citep{milne_2021} and 38 for failed web-camera-checks. To ensure data quality and maximize retained participants, six simple attention checks occured during each of the training segments and test segments. These attention checks were not in \cite{nixon2020mice} and were only added after early pilot data indicated many participants were not paying attention during training phase. That is, we nee

Three were removed for being below the two MAD \citep{Leys_2013} range in overall speech cue sensitivity. After removal, 47 participants' (age: $\mu$ = 34.5 years, \textit{sd} = 9.6 years) data were retained for analysis. In total, the tasks reported here took participants approximately 30 minutes to complete. All participants were paid for their time. The study was approved by the authors' Institutional Review Board.
