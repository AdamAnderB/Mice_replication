\section{Abstract}

This study replicates and extends the second unlearning experiment, originally developed to investigate error-driven learning in speech sound acquisition (Nixon, 2020). We adapt the experiment to include Japanese and Mandarin, exploring how prediction errors shape learning in typologically distinct language systems. The eye-tracking extension allows us to operationalize prediction error through participants’ gaze patterns during training, providing real-time data on how discrepancies between predicted and actual outcomes influence learning. This novel approach captures the dynamic feedback loop between prediction error and learning in real-time, contributing to a deeper understanding of adaptive processes during speech acquisition.

This study fits within the broader literature that challenges the view of language learning as purely statistical tracking of speech cues. Instead, error-driven models, such as the Rescorla-Wagner model, suggest that learners adjust based on feedback from prediction errors, rather than passively absorbing cue frequencies. Nixon (2020) demonstrated that both cue competition and unlearning from prediction error are critical components of speech sound acquisition. By extending this paradigm to Japanese and Mandarin, this study provides new insights into cross-linguistic variation in the role of prediction error during language learning, particularly in how pitch and segmental contrasts are processed differently across these languages.


\section{Keywords}
Error-Driven Learning, Prediction Error, Cross-Linguistic Study, Japanese, Mandarin, Southern Min, Eye-Tracking
\newpage


