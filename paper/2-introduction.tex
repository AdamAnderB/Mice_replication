\section{1. Introduction}

\section{Introduction}


Second language speech learning is multidimensional and complex, with each language posing its own unique difficulties for learners. For example, E



Early research took this complexity to imply language specific mechanisms for language acquisition. Meaning that language is learned with mechanisms that are unique to language learning alone. 


More recently, however, general learning mechanisms have been used   


, particularly the acquisition of phonological contrasts, is shaped by mechanisms that allow learners to adjust their predictions based on feedback from the environment. One prominent theory, error-driven learning, posits that learners refine their knowledge through prediction errors — the discrepancy between expected and actual outcomes. This theory stands in contrast to statistical learning models, which emphasize frequency and co-occurrence of cues. The role of prediction error in speech sound acquisition has gained attention, offering insights into how learners recalibrate their understanding of speech contrasts over time.

\subsection{Nixon’s Study on Error-Driven Learning}

Nixon (2020) provides a compelling case for error-driven learning in speech acquisition, focusing on tonal contrasts in Southern Min. In this study, Nixon demonstrates that learners update their internal models not simply by tracking the frequency of speech cues, but through feedback-driven adjustments when their predictions about speech sounds fail. The study showed how cue competition and unlearning shape speech sound acquisition, challenging purely statistical models. Nixon's work highlights how prediction errors, rather than exposure alone, drive learning, particularly in tonal contrasts, where acoustic differences are often subtle yet critical for meaning.

\subsection{Expanding Beyond Tones: Focus on Consonants and Vowels}

While Nixon's work focused on tonal contrasts, the current study extends the framework of error-driven learning to consonantal and vocalic contrasts. By examining the *zh-j* fricative distinction in Mandarin and vowel duration contrasts in Japanese, we aim to explore how prediction errors shape the learning of these different phonetic dimensions. These distinctions offer a more nuanced view of error-driven learning, as consonantal and vocalic contrasts often require different acoustic sensitivities compared to tones. By focusing on these contrasts, we seek to understand how learners adjust to more subtle differences in speech, broadening the scope of error-driven learning beyond tonal languages.

\subsection{Language-Specific Comparisons: Why Mandarin and Japanese?}

\subsubsection{Mandarin Fricative Comparison: zh-j}

Mandarin presents a unique phonological structure, particularly with the *zh-j* fricative contrast, which is differentiated by subtle acoustic features. Learners must rely on fine distinctions in place of articulation to correctly identify these sounds, making it an ideal case for studying prediction error. Understanding how prediction errors guide learners to correctly differentiate these fricatives offers a deeper look into the process of phonological learning.

\subsubsection{Japanese Vowel Duration: kuti vs. kuuti}

In Japanese, vowel duration plays a critical role in lexical meaning, as exemplified by words like *kuti* (mouth) and *kuuti* (air). The ability to distinguish between short and long vowels is crucial for accurate speech perception and production. Examining how prediction errors influence the learning of vowel length distinctions provides insight into how learners adjust their perception of temporal features in speech, further expanding our understanding of error-driven learning in vocalic contrasts.

\subsection{Rationale for the Eye-Tracking Extension}

In addition to replicating the unlearning experiment across Mandarin and Japanese, we incorporate an eye-tracking extension to operationalize prediction errors during training. Eye-tracking allows us to observe how learners' gaze patterns reflect their real-time adjustments to prediction errors. By tracking visual attention during speech tasks, we can measure the immediate feedback-driven learning process and observe how prediction errors are corrected over time.

\subsection{Contributions to the Field}

This study builds on Nixon's foundational work, extending error-driven learning to consonants and vowels in Mandarin and Japanese. By focusing on fricatives and vowel duration, this research deepens our understanding of how prediction errors function across different phonetic dimensions. The addition of eye-tracking further enhances our ability to capture the dynamic process of prediction error correction, offering a novel perspective on real-time learning in speech acquisition.

\section{Research Questions and Hypotheses}

\subsection{Research Questions}

\begin{itemize}
    \item How do prediction errors influence the learning of the *zh-j* fricative distinction in Mandarin?
    \item How do prediction errors shape learners' ability to differentiate vowel lengths (*kuti-kuuti*) in Japanese?
    \item Does eye-tracking provide evidence of distinct prediction error processes for consonants and vowels?
\end{itemize}

\subsection{Hypotheses}

\begin{itemize}
    \item Prediction errors will facilitate the learning of the *zh-j* fricative distinction in Mandarin, leading to improved categorization over time.
    \item Learners will adjust their perception of vowel length in Japanese through prediction errors, improving their ability to distinguish between short and long vowels.
    \item Eye-tracking data will reveal different patterns of prediction error correction for consonants and vowels, reflecting the varying cognitive demands of these phonetic contrasts.
\end{itemize}